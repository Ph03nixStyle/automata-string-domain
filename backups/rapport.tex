\documentclass{article}

\title{Étude par interprétation abstraite de la correction de strings construits dynamiquement}
\author{Maire Grégoire}
\date{}

\usepackage{dsfont}
\usepackage{url}
\usepackage{hyperref}
\usepackage{amssymb}
\usepackage{amsthm}
\usepackage{amsmath}
\usepackage{enumitem}
\usepackage{algorithmic}
\usepackage{algorithm}

\usepackage[style=0,ntheorem]{mdframed}
\mdfsetup{%
topline=false,
rightline=false,
bottomline=false,
linewidth=1pt,
innertopmargin=0pt,
innerbottommargin=5pt,
innerleftmargin=10pt,
innerrightmargin=0pt,
skipabove=\medskipamount,
skipbelow=\medskipamount
}

\newenvironment{preuve}{\begin{proof}[Preuve]}{\end{proof}}
\newtheorem{theo}{Théorème}
\newtheorem{proposition}{Proposition}
\newtheorem{corollaire}{Corollaire}
\newtheorem{algo}{Algorithme}
\newtheorem{definition}{Définition}
\newtheorem{lemme}{Lemme}
\newtheorem{exemple}{Exemple}[section]
\newenvironment{mdtheo}[1][]%
   {\begin{mdframed}[linecolor=black]\begin{theo}[#1]}
   {\end{theo}\end{mdframed}}
\newenvironment{mdlemme}[1][]%
   {\begin{mdframed}[linecolor=orange]\begin{lemme}[#1]}
   {\end{lemme}\end{mdframed}}
\newenvironment{mddefinition}[1][]%
   {\begin{mdframed}[linecolor=black]\begin{definition}[#1]}
   {\end{definition}\end{mdframed}}
\newenvironment{mdalgo}[1][]%
   {\begin{mdframed}[linecolor=black]\begin{algo}[#1]}
   {\end{algo}\end{mdframed}}

\newcommand{\prog}[1]{{\langle #1 \rangle}} 


\begin{document}

\maketitle

\section{Bases de l'interprétation abstraite}
\subsection{Introduction}
L'interprétation abstraite est une branche de l'informatique qui a pour but d'inférer
automatiquement des invariants quelconques sur un programme, et ce, à l'aide d'approximations \cite{CousotCousot77-1}.
Nous nous concentrerons ici en particulier sur l'inférence d'invariants sur les valeurs prises par
une variable de type \texttt{string} en un point quelconque d'un programme.

Nous justifions ici la nécessité d'approximations lors de telles analyses par le théorème de Rice \cite{Rice},
qui détermine l'indécidabilité d'une telle tâche:

\begin{mdtheo}[Théorème de Rice]
    Soit f une fonction booléenne totale, dont l'entrée est le code source d'un algorithme. On suppose que:
    \begin{itemize}
        \item f est non-constante;
        \item f conserve l'équivalence sémantique : si $A$ et $B$ sont deux programmes
        sémantiquement équivalents, noté $A \sim B$, alors $f(A) = f(B)$.
    \end{itemize}
    Alors son problème de décision associé est \underline{indécidable}.
\end{mdtheo}

\begin{preuve}
    Soient les programmes $\prog{F} = \prog{\texttt{while (true)}}$,
    et $\prog{B}$ tel que $f(\prog{B}) \neq f(\prog{F})$ \\
    On pose $tr$: $(\prog{A},e) \longmapsto \prog{x \mapsto A(e) \;;\; B(x)}$ \\
    Soient $A$ un algorithme et $e$ une entrée de $A$.
    \begin{itemize}
        \item \underline{Si $A(e)$ finit:} $tr(\prog{A}, e)$ $\sim$ $\prog{B}$ donc $f(tr(\prog{A}, e)) = f(\prog{B})$
        \item \underline{Sinon:} $tr(\prog{A}, e) \sim \prog{F}$ donc $f(tr(\prog{A}, e)) = f(\prog{F})$
    \end{itemize}    
    On a ainsi:
    \begin{align*}
      f(tr(\prog{A}, e)) = f(\prog{B}) & \iff A(e) \text{ termine} \\
                                       & \iff Arret(\prog{A}, e)
    \end{align*}
  
    D'où une réduction du problème $Arret$ (resp. $\neg Arret$), d'où l'indécidabilité vu $Arret$ indécidable.
\end{preuve}

% TODO inclure exemple de référence modulo 3

\subsection{Domaine concret et domaine abstrait d'étude}

On définit par la suite les objets avec lesquels on travaillera afin de préciser ce que l'on entend par
\emph{approximations}. 

\begin{definition}[Treillis] % TODO pas certain d'inclure : Sinon on va me poser des questions dessus
    Un treillis est un ensemble muni d'une relation d'ordre partielle, d'un plus grand et plus petit élément,
    et dans lequel chaque paire d'élément admet une borne inférieure et une borne supérieure.
    Ici, nos bornes supérieures seront des unions, et nos bornes inférieures des intersections.
\end{definition}

On peut formaliser le processus d'abstraction à l'aide de deux \emph{domaines}, qui sont en fait des treillis.
On définit tout d'abord un \emph{domaine concret} qui permet de représenter les états dans lesquels peuvent
se trouver une variable à un instant donné du programme, puis un \emph{domaine abstrait}
qui sur-approxime les valeurs concrètes, levant par la même occasion l'indécidabilité.

\begin{definition}[Domaine concret]
    Soit $\Sigma$ un alphabet.
    Un élément du domaine concret représente l'ensemble exact des valeurs que peut prendre une variable
    en un point d'un programme. Notre domaine concret d'étude sera celui des variables de type
    \emph{\texttt{string}}, $\mathcal{P}(\Sigma^*)$, ou plutôt son treillis associé :
    $(\mathcal{P}(\Sigma^*), \subseteq, \emptyset, \Sigma^*, \cup, \cap)$
\end{definition}

\begin{definition}[Domaine abstrait]
    Un élément du domaine abstrait représente une sur-approximation d'un élément du domaine concret.
    Nous nous intéresserons au domaine abstrait des automates. Ainsi, notre domaine abstrait sera
    le treillis des automates, muni de l'union et l'intersection d'automates, de l'inclusion d'un
    langage induit par rapport à un autre, de l'automate vide et de l'automate complet.
\end{definition}

Une analyse de programme sur une variable va donc renvoyer un élément du domaine abstrait, qui
sur-approximera les valeurs concrètes que peut prendre cette variable.
On attend en particulier d'un domaine abstrait que ses éléments soient représentable par machines,
ce qui est le cas des automates. Ces derniers permettent de représenter une grande classe de langages
(les langages rationnels), et il est toujours possible de sur-approximer tout langage par un langage rationnel,
dans le pire des cas en considérant le langage $\Sigma^*$ comme sur-approximation. \\
D'autres domaines abstraits sur les variables \texttt{string} existent dans la littérature, mais
à ma connaissance aucun n'est directement implémenté sur des automates de la sorte.
Parmis les domaines existants, on peut par exemple trouver un domaine sur les préfixes communs des mots \cite{string_analysis}. % TODO peut etre changer


\section{Preuves de programmes}
\subsection{Récolte de l'ensemble des états de la variable d'étude}

On considère un langage de programmation simple, auquel est intégré un certain nombre de primitives
n'ayant qu'une seule variable, de type \texttt{string}, et ayant un certain nombre de primitives agissant sur cette dernière.
On en présente la syntaxe en annexe. On ajoute la possibilité d'avoir une condition
\texttt{Unknown}, représentant par exemple une exécution non-déterministe ou dépendant d'un programme externe.

% TODO annexe rajouter récolte d'états


% TODO réorganiser ce qui précède pour metttre + en valeur


\begin{proposition}[Récolte d'états du \texttt{while}]
    Soit un programme \texttt{P1; (while Unknown do P2)}, tel que \texttt{P2} ne change le string d'étude qu'à travers
    des concaténations droites. Soit $\mathcal{A}_1$ l'automate résultant de l'analyse de \texttt{P1} (resp. $\mathcal{A}_2$).
    Alors, sans considération aucune de la condition réelle \texttt{Unknown}, l'automate suivant reconnaîssant
    $L(\mathcal{A}_1) . (L(\mathcal{A}_2)) ^*$ reconnaît \emph{au moins} l'ensemble des strings qui pourront être construit
    par le programme.
\end{proposition}

On peut de la même manière définir une sémantique de récolte de l'ensemble des états possibles
dans un programme écrit dans le langage précédemment décrit, faisant usage d'unions, intersections,
et concaténations d'automates. J'ai ainsi implémenté une librairie complète sur les automates,
ainsi qu'une application de la récolte d'états sur le langage. Ce code est en annexe.

\begin{definition}[Correction d'une analyse]
    On dit dit d'une analyse automatique d'invariant qu'elle est \emph{correcte}
    si sa valeur de retour est inclue dans l'ensemble des valeurs satisfaisant l'invariant.
    Dans notre cas d'étude, on souhaite que l'état du domaine abstrait (un automate) en fin
    d'analyse de programme soit inclu dans un automate de référence, fourni par l'utilisateur.
\end{definition}

En effet, toute analyse renvoyant une sur-approximation de l'ensemble des valeurs
qui peuvent être prises par la variable d'étude en fin de programme,
il suffit que cette sur-approximation soit contenue dans l'ensemble des valeurs acceptables pour cette variable
pour que l'on ait alors une preuve rigoureuse de la correction du programme.
On remarque qu'il est ainsi possible que les approximations effectuées soient trop larges et qu'on puisse alors
déterminer dans l'analyse qu'un programme est faux, alors qu'il est en réalité correct. Le but est alors de
minimiser ces faux-positifs.

\begin{proposition}[Complétude du système de récolte]
    Pour une expression rationnelle $e$ donnée, il est possible de générer un automate
    pour lequel l'analyse génèrera un automate équivalent à $e$.
    On crée ainsi une manière de construire des expressions rationnelles à partir d'un programme.
\end{proposition}

\begin{preuve}
    Il suffit d'observer que l'on peut effectuer avec la récolte d'états tout ce qui compose
    un langage rationnel dans sa définition inductive: le langage vide, les lettres $a \in \Sigma$,
    une union par un \texttt{if/else}, une concaténation par un \texttt{push\_right(str)}, et une étoile de Kleene par un \texttt{while}.
\end{preuve}

\begin{proposition}[Génération de programme]
    À l'inverse, on peut générer un programme à partir d'une expression rationnelle $e$, où l'on garantit que
    ce programme ne pourra créer que des strings appartenant au langage $L(e)$. L'utilisateur n'a alors plus qu'à modifier les
    conditions \texttt{Unknown} pour obtenir un programme correct.
\end{proposition}

\begin{preuve}
    On peut construire le programme inductivement sur la structure de $e$.
    \begin{itemize}
        \item Si $e = \varepsilon$, alors \texttt{newstring()} convient;
        \item Si $e = a \in \Sigma$, alors \texttt{push\_right(a)} convient;
        \item Si $e = e_1.e_2$ avec \texttt{P1} et $\mathcal{P}_2$ les programmes de $e_1$ et $e_2$, alors \texttt{P1; P2} convient;
        \item Si $e = e_1 \;|\; e_2$, alors \texttt{if (Unknown) then P1 else P2} convient;
        \item Si $e = e_1^*$, alors \texttt{while (Unknown) do P1} convient;
    \end{itemize}
\end{preuve}

\section{Optimisations de l'implémentation pratique}

\subsection{Explosion exponentielle du nombre d'états de l'automate}

\begin{proposition}[Union exponentielle]
    Soit $\mathcal{A}$ l'automate entrant dans une branche \texttt{if}/\texttt{else}, $\mathcal{A'}$
    l'automate résultant. Si aucune des deux branches possibles n'a de primitive d'effacement de lettres sur $\mathcal{A}$,
    alors $\mathcal{A'}$ a au moins 2 fois plus d'états que $\mathcal{A}$.
\end{proposition} 

\begin{preuve}
    La preuve est immédiate de par l'action d'union d'automates "mis côtes à côtes" qui duplique
    les états de l'automate initial.
\end{preuve}

On a un doublement d'état, or c'est pas toujours nécessaire de vraiment doubler les états,
comme le montre l'exemple suivant:
% TODO montrer programme explosion_etats1 pour montrer qu'on peut réduire exponentiellement, pour justifier ce qui suit.

Pour parier à ce problème, on introduit la notion d'automate minimal à l'aide des propositions suivantes.

\begin{definition}[Automate, DFA]
    On considère par la suite un automate $\mathcal{A} = (Q, \Sigma, q_0, F)$.
\end{definition}


\begin{definition}[Résiduel]
    On appelle \emph{résiduel} ou \emph{quotient à gauche} d'un langage $L$ tout langage
    de la forme $u^{-1}L = \{v \in \Sigma^*, u.v \in L\}$ \\
    L'ensemble des résiduels d'un langage L s'écrit alors $res(L)$.
\end{definition}


\begin{proposition}
    Pour $\mathcal{A}$ complet reconnaissant L
    et tel que tous ses états soient accessible, on a
    $\{L_q, q \in Q\} = res(L)$
\end{proposition}

\begin{preuve}
    \fbox{$\subseteq$} : Soit $ L_q$ pour $q \in Q$. Vu $q$ accessible, il existe $w \in \Sigma^*$
    tel que $q_{0}.w = q$. Alors:
    \begin{align*}
        L_q & = \{u \;|\; q.u \in F\} \\
            & = \{u \;|\; (q_0.w).u \in F\} \\
            & = \{u \;|\; w.u \in L\} \\
            & = w^{-1}L
    \end{align*}
    D'où l'inclusion.
    
    \fbox{$\supseteq$} : Soit $u^{-1}L \in res(L)$. Vu $\mathcal{A}$ complet, soit
    $q = q_{0}.u$. On a alors $u^{-1}L = L_q$. D'où l'inclusion réciproque.
\end{preuve}

\begin{corollaire}
    Tout langage rationnel a un nombre fini de résiduels,
    et tout automate déterministe reconnaissant L possède au moins $|res(L)|$ états.
\end{corollaire}

\begin{definition}[Congruence de Nérode]
    La congruence de Nérode est une relation d'équivalence sur les états de $\mathcal{A}$
    définie par: pour tout $q, q' \in Q$, on a $[q] = [q'] \Leftrightarrow L_q = L_{q'}$.
\end{definition}

% TODO rajouter une transition pour l'existence de l'automate minimal etc

\begin{mdalgo}[Algorithme de minimisation de Brzozowski] % TODO regarder autre algo itératif pour trouvere les classes
    Soient:
    \begin{itemize}[topsep=0pt, itemsep=0pt]
        \item $d(\mathcal{A})$ la partie accessible du déterminisé "par parties" de $\mathcal{A}$
        \item $t(\mathcal{A})$ le transposé de $\mathcal{A}$.
    \end{itemize} \hfill\break
    Alors $d(t(d(t(\mathcal{A}))))$ est l'automate minimal de $\mathcal{A}$.
\end{mdalgo}

\begin{preuve}
    Cet automate reconnaît le même langage que $\mathcal{A}$ sans soucis. \\
    De plus, soient $P$ et $P'$ deux états distincts de $\mathcal{A}'$.
    Par définition, $P$ et $P'$ sont deux ensembles-états de l'automate $d(t(\mathcal{A}))$.
    Soit $R$ un état de $d(t(\mathcal{A}))$, tel que $R \in P$ et $R \notin P'$. $R$ est aussi un ensemble-état, qui vérifie,
    par définition du transposé, $R = \{q \;|\; q \xrightarrow{w} f \text{ dans } \mathcal{A} \text{, avec } f \in F\}$ pour un certain mot $w$.
    On en déduit que dans $\mathcal{A'}$, il existe un chemin étiquetté par $w$ vers un état final,
    alors que ce n'est pas le cas pour $P'$. Donc $L_P \neq L_{P'}$ et $P$, $P'$ ne sont pas équivalents.
\end{preuve}


% TODO inclure graphique d'évolution d'états: on est content !
% TODO code candidat sur diapo tétraconcours
Malheureusement, il arrive que cette minimisation ne soit pas suffisante.


\subsection{Explosion déterministe du nombre d'états de l'automate}

Considérons le programme [...]. Son analyse génère le langage décrit ci-dessous:
% TODO montrer programme explosion_etats2 avec les langages An.

\begin{proposition}
    Tout automate déterministe $\mathcal{A}_n$ reconnaissant le langage défini par "l'ensemble des mots ayant la lettre
    \emph{\texttt{a}} $n$ lettres avant la fin" possède au moins $2^n$ états.
\end{proposition}

\begin{preuve}
    Supposons par l'absurde disposer d'un automate déterministe $\mathcal{A}_n$
    reconnaissant ce langage et possédant $|Q| < 2^n$ états. \\
    Soit $M = \Sigma^n$. Soit $u \in M$. On peut lire $u$ dans $\mathcal{A}_n$, car le mot $u.0^{n+1}$
    est accepté par l'automate et son préfixe $u$ n'est donc pas bloquant. On peut ainsi lire tout mot de $M$ dans $\mathcal{A}_n$.
    De plus, on a $|M| = 2^n > |Q|$, donc par lemme des tiroirs: il existe $m, m' \in M$ et $q \in Q$ tel que $q_0.m = q_0.m' = q$
    et $m \neq m'$. On pose $m = m_1...m_n$ (resp. $m'$). Soit $k$ tel que $m_k \neq m'_k$. On pose $m_k = a$ et $m'_k \neq a$ (sans perte
    de généralité). On a donc $u = m_1...m_k.0^n$ accepté par $\mathcal{A}_n$, et $v = m_1...m_k.0^n$ non-accepté.
    Or, $\mathcal{A}_n$ est déterministe, donc $q.0^n$ ne peut amener que dans un seul état. Il vient que $q_0.u = q_0.v$.
    Finalement, cet état est à la fois final et non-final. On obtient une absurdité.
\end{preuve}

Il est pourtant très facile d'écrire un automate non-déterministe reconnaissant ce langage
avec une croissance linéaire en n.  \\ % TODO inclure image
Une solution consisterait alors à trouver un automate minimal non-déterministe, qui permet une représentation
(exponentiellement) plus compactes. Je n'ai pas eu le temps de me pencher sur cette question,
les algorithmes permettant une telle minimisation étant très complexes.

\section{Justification du choix de l'object d'étude}

L'étude a ici été limitée aux langages réguliers, et non pas aux langages algébriques. En effet,
le problème d'une inclusion dans une autre est indécidable dans le cas général, ce qui rend impossible la vérification
de la correction d'une analyse. On le montre ici par le biais de plusieurs propositions préliminaires.
On admet tout d'abord que le problème de Post, défini ci-après, est indécidable:

\hfill\newline
\textbf{CorrespondancePost} \\
\underline{Entrée:} Des couples de mots $\{(u_1, v_1), ..., (u_n, v_n)\}$ \\
\underline{Sortie:} Vrai s'il existe une suite
    $(i_p)_{p \in [1, m]}$ avec $u_{i_1}...u_{i_m} = v_{i_1}...v_{i_m},$ Faux sinon \newline

\begin{proposition}
    Le problème suivant est indécidable: \\
    \underline{Entrée:} $G$ une grammaire non-contextuelle \\
    \underline{Sortie:} Vrai si $L(G) = \Sigma^*$, Faux sinon
\end{proposition}

\begin{preuve}
    Pour une entrée donnée de Post, on fixe le langage $L_u$ (resp. $L_v$) défini
    sur l'alphabet $\Sigma_1 = \Sigma \cup A$ avec $A = \{a_1, ..., a_n\}$ des nouvelles lettres, par la grammaire suivante:
    $S \longrightarrow u_1 S a_1 \;|\; ... \;|\; u_n S a_n$ (de même pour $L_v$). Ces deux grammaires sont
    non-ambigües, et on a les équivalences suivantes:
    \begin{align*}
        \text{Il n'existe pas de solution à l'entrée de Post} & \iff L_u \cap L_v = \emptyset \\
                                                              & \iff (L_u \cap L_v)^c = \Sigma_1^* \\
                                                              & \iff L_u^c \cup L_v^c = \Sigma_1^*
    \end{align*} 
    Où l'on aura noté $L^c$ le complémentaire de $L$, et où on admet que
    $L_u^c$ et $L_v^c$ sont encore algébriques (donc leur union l'est également).
    D'où une réduction du problème de Post à notre problème avec $G = L_u^c \cup L_v^c$ et $\Sigma = \Sigma_1$, d'où l'indécidabilité.
\end{preuve}

\begin{corollaire}
    Le problème suivant est indécidable: \\
    \underline{Entrée:} $G$, $G'$ des grammaires non-contextuelles \\
    \underline{Sortie:} Vrai si $L(G') \subseteq L(G)$, Faux sinon
\end{corollaire}

\begin{preuve}
    Il suffit de remarquer dans la proposition qui précède que $\Sigma^* = L(G)$ est équivalent
    à $\Sigma^* \subseteq L(G)$.
\end{preuve}

Ce corollaire achève de démontrer qu'il aurait été impossible d'appliquer sur les grammaires
la même méthode de vérification par inclusion dans une langage de référence que les automates.


\section{Conclusion} % TODO réécrire la conclusion lol
Nous avons construit un domaine fonctionnel qui permet de reconnaître automatiquement certaines propriétés vérifiées par un string,
tout en s'inquiétant de l'efficacité de l'implémentation pour éviter une explosion de complexité.
Tout est une question de compromis, nous avons ici construit un domaine coûteux en temps d'analyse
comparé à un simple domaine préfixe par exemple, mais qui permet une précision bien supérieure
sur les propriétés du string construit. Ça dépend évidemment des propriété qu'on veut vérifier.
Autres travaux existant: domaine qui prend un automate de référence et fait avancer l'automate au fil du programme.
Futur travail: [...].

\bibliographystyle{plain}
\bibliography{../bibliographie/bibliographie}


\section{Annexe}
% Mettre les programmes auxquels je fais mention + haut ici.
% Programme modulo 3 exemple.




\end{document}